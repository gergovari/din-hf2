\section{Súlypontok gyorsulása}

Az előző házi feladatban már ki lett számolva ez a két gyorsulás.

\begin{align}
	&\pmb{a}_\text{S1} 
        = \begin{bmatrix}
                \aSonex \\ \aSoney \\ \aSonez
        \end{bmatrix} \siunit{}{\meter\per\second^2} \\
	&\pmb{a}_\text{S2} 
        = \begin{bmatrix}
                \aStwox \\ \aStwoy \\ \aStwoz
        \end{bmatrix} \siunit{}{\meter\per\second^2}
\end{align}

A harmadik pedig egyszerűen megkapható az említett dokumentumban levő levezetésből.

\begin{equation}
	\pmb{a}_\text{S3} 
	= \pmb{a}_\text{A} 
	= \pmb{a}_\text{E} + \pmb{\epsilon}_3 \times \pmb{r}_\text{EA} - \omega_3^2\pmb{r}_\text{EA}
        = \begin{bmatrix}
                \aStrx \\ \aStry \\ \aStrz
        \end{bmatrix} \siunit{}{\meter\per\second^2} 
\end{equation}
