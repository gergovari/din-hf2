\section{Tömítés kiválasztása}

\subsection[Minimális tömítőerő]{Minimális tömítőerő\protect\footnote{A feladathoz mellékelt segédletből származó számítások. (7. oldal)}}

A belső nyomás miatti csőerő hat ellen az üzemi nyomásnak. A gyűrűfelületi csőerő nyom ellen a gyűrű alsó felülete alá benyomódó folyadéknak. A minimális tömítő erő szükséges ahhoz hogy a tömítetség kialakuljon. Ezek összege adja a csavarra ható üzemi erőt.

\begin{align}
	&z = \frac{{d_2}_t - {d_1}_t}{2} = \siunit{\tomz}{db} \\
	&b_t^* = 9 + 0.2z = \siunit{\tombtstar}{\mm}
\end{align}

\begin{align}
	&F_\text{cső} 
	= \frac{\text{DN}^2 \pi}{4} p_\text{ü} = \n{\tomcso} \\
	&F_\text{p} 
	= \frac{\left(d_t^2 - \text{DN}^2\right)\pi}{4} p_\text{ü} 
	= \n{\tomp} \\
	&F_\text{töm} = n_t p_\text{ü} \pi d_t b_t^* = \n{\tomtu}
\end{align}

\begin{equation}
	F_\text{csavar üzemi} 
	= F_\text{cső} + F_\text{p} + F_\text{töm} 
	= \n{\csavaruzemi}
\end{equation}

\begin{align}
	&{n_\text{bizt}}_t = \siunit{\csavarn}{-} \\
	&F_\text{csavar szerelési} 
	= {n_\text{bizt}}_t F_\text{csavar üzemi}
	= \n{\csavarszerel}
\end{align}

\begin{center}
	\begin{tabular}{l}
		$z$: fogak száma \siunit{}{db} \\
		$b_t^*$: tömítés hatásos szélessége\protect\footnote{Képlete a feladathoz mellékelt segédletből. (8. oldal, 3. táblázat)} \siunit{}{\mm} \\
		$F_\text{cső}$: belső nyomásból származó csőerő \siunit{}{\newton} \\
		$F_\text{p}$: belső nyomásból származó gyűrűfelületi erő \siunit{}{\newton} \\
		$F_\text{töm}$: minimális tömítő erő \siunit{}{\newton} \\
		$F_\text{csavar üzemi}$: csavarokra ható üzemi erő \siunit{}{\newton} \\
		${n_\text{bizt}}_t$: csavarokra ható szerelési erőhöz választott biztonsági tényező \siunit{}{-} \\
		$F_\text{csavar szerelési}$: csavaroknál alkalmazott szerelési erő \siunit{}{\newton} \\
	\end{tabular}
\end{center}

\newpage
\subsection{Szabvány -és anyagválasztás}
A DIN EN 1514-6 B29A PN100 szabvány lett választva  és ez a tömítés nagy nyomásokat is kibír. 1.4541 fémből és egy PTFE borításból készül ahol a fém fésük deformálják a műanyagot az előfeszítés hatására ezzel előidézve a tömítőerőt.

\subsection[Előterv]{Előterv\protect\footnote{Előterv a DIN EN 1514-6 B29A PN100 szabvány alapján.}}
\begin{figure}[hbt!]
	\centering
	\includegraphics[scale=.34]{./images/tomites.png}
	\caption{Tömítés előtervének rajza}
\end{figure}
\begin{align*}
	&{d_1}_t = \siunit{\tomdone}{\mm} \\
	&{d_2}_t = \siunit{\tomdtwo}{\mm} \\
	&{d_3}_t = \siunit{\tomdthree}{\mm} \\
	&b_t = \siunit{\tombt}{\mm} \\
	&b_m = \siunit{\tombm}{\mm} \\
	&h_{\substack{\text{min}\\\text{max}}} = \substack{\siunit{\tomhmin}{\mm}\\\siunit{\tomhmax}{\mm}} \\
\end{align*}
\begin{center}
	\begin{tabular}{l}
		${d_1}_t$: tőmítés belső átmérő \siunit{}{\mm} \\
		${d_2}_t$: tőmítés felfekvő felület külső átmérő \siunit{}{\mm} \\
		${d_3}_t$: távtartó gyűrű külső átmérő \siunit{}{\mm} \\
		$b_t$: távtartó gyűrű vastagság \siunit{}{\mm} \\
		$b_m$: fém mag magasság \siunit{}{\mm} \\
		$h_{\substack{\text{min}\\\text{max}}}$: szerelés utáni/előtti távolsága \\PTFE lemezeknek a vasmag tetejétől \siunit{}{\mm} \\
	\end{tabular}
\end{center}
