\section{$F_y$ és a reakcióerők}

A fenti egyenletrendszer megoldásából ezek kaphatóak.

\begin{align}
	&{\theta_\text{S1}}_\text{z} = \frac{1}{2} m_1 l_1^2 \\
	&{\theta_\text{S2}}_\text{z} = \frac{1}{12} m_2 l_2^2 \\
	&{\theta_\text{S3}}_\text{z} = \frac{1}{2} m_3 l_3^2 \\
\end{align}

\begin{align}
	&\pmb{C}
	= \begin{bmatrix}
		\Cx \\ \Cy \\ 0
	\end{bmatrix} \siunit{}{\newton} \\
	&\pmb{B}
	= \begin{bmatrix}
		\Bx \\ \By \\ 0
	\end{bmatrix} \siunit{}{\newton} \\
	&\pmb{F}
	= \begin{bmatrix}
		0 \\ \Fy \\ 0
	\end{bmatrix} \siunit{}{\newton} \\
\end{align}

\begin{align}
	&\pmb{E}
	= \begin{bmatrix}
		\Ex \\ 0 \\ 0
	\end{bmatrix} \siunit{}{\newton} \\
	&\pmb{D}
	= \begin{bmatrix}
		\Dx \\ 0 \\ 0
	\end{bmatrix} \siunit{}{\newton} \\
	&\pmb{N_1}
	= \begin{bmatrix}
		0 \\ \None \\ 0
	\end{bmatrix} \siunit{}{\newton} \\
	&\pmb{N_3}
	= \begin{bmatrix}
		0 \\ \Ntr \\ 0
	\end{bmatrix} \siunit{}{\newton}
\end{align}
