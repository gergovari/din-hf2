\section{Kinetikus energia és erőrendszer teljesítménye}

\subsection{Kinetikus energia}
\begin{equation}
	E_k
	= \sum_{i=1}^3\frac{1}{2} m_i v_\text{Si}^2 + \frac{1}{2} {\theta_\text{Si}}_z \omega_i^2
	= \siunit{\Ek}{\joule}
\end{equation}

\subsection{Teljesítmény}
\begin{equation}
	P
	= \sum_{i=1}^3 \left[
		\sum_{j=1}^{j_\text{max}(i)} \left(
			\pmb{F_j} \cdot \pmb{v_Si}
		\right) + \pmb{\theta_\text{Si}} \cdot \pmb{\omega_i} \cdot \pmb{\epsilon_i}
	\right]
	= \siunit{\P}{\watt}
\end{equation}

\subsection{Teljesítménytétel}

A két módon számolt teljesítmény megegyezik, tehát a teljesítménytétel teljesül.

\begin{align}
	\dot{E}_k
	&= \sum_{i=1}^3
		m_i \left( \pmb{v_\text{Si}} \cdot \pmb{a_\text{Si}} \right)
		+ {\pmb{\theta_\text{Si}}} \cdot \pmb{\omega_i} \cdot \pmb{\epsilon_i}
	= \siunit{\Tdot}{\watt} \\
	\dot{E}_k
	&= P
\end{align}
